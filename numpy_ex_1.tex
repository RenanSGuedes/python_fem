\documentclass{article}%
\usepackage[T1]{fontenc}%
\usepackage[utf8]{inputenc}%
\usepackage{lmodern}%
\usepackage{textcomp}%
\usepackage{lastpage}%
\usepackage{amsmath}%
%
%
%
\begin{document}%
\normalsize%
\section{Método dos elementos finitos: Treliça Plana}%
\label{sec:MtododoselementosfinitosTreliaPlana}%
\subsection{Introdução}%
\label{subsec:Introduo}%
O método dos elementos finitos faz uso das matrizes de rigidez como forma de análise de um sistema. Dessa forma, tomando como base o sistema global de coordenadas é possível saber por meio das características geométricas e propriedades materiais dos elementos os esforços e deformações em cada um. Todavia, dentre as propriedades mencionadas anteriormente existe o módulo de Young ($E$), área da seção transversal das barras $A$, e o comprimento das barras $L$. Como a análise feita é em relação ao sistema global de coordenadas, calcula-se o ângulo $\theta$%
\[%
k_{n}=\dfrac{EA}{L} \begin{bmatrix}%
\cos^2\theta&\cos\theta\sin\theta&-\cos^2\theta&-\cos\theta\sin\theta\\%
\cos\theta\sin\theta&\sin^2\theta&-\cos\theta\sin\theta&-\sin^2\theta\\%
-\cos^2\theta&-\cos\theta\sin\theta&\cos^2\theta&\cos\theta\sin\theta\\%
-\cos\theta\sin\theta&-\sin^2\theta&\cos\theta\sin\theta&\sin^2\theta%
\end{bmatrix}%
\]

%
\subsection{Componentes do sistema}%
\label{subsec:Componentesdosistema}%
Número de barras da estrutura: 4\newline%
%
Número de nós da estrutura: 4%
\subsubsection{Comprimento das barras}%
\label{ssubsec:Comprimentodasbarras}%
Barra 1: 9.0 cm\newline%
%
Barra 2: 15.0 cm\newline%
%
Barra 3: 9.0 cm\newline%
%
Barra 4: 12.0 cm

%
\subsubsection{Módulo de elasticidade das barras (MPa)}%
\label{ssubsec:Mdulodeelasticidadedasbarras(MPa)}%
E = 30.0 MPa

%
\subsection{Matrizes de rigidez dos elementos}%
\label{subsec:Matrizesderigidezdoselementos}%
Elemento {1}:%
\[%
k_{1}= \begin{bmatrix}%
k_{11}^{(1)}&k_{12}^{(1)}&k_{13}^{(1)}&k_{14}^{(1)}\\%
k_{21}^{(1)}&k_{22}^{(1)}&k_{23}^{(1)}&k_{24}^{(1)}\\%
k_{31}^{(1)}&k_{32}^{(1)}&k_{33}^{(1)}&k_{34}^{(1)}\\%
k_{41}^{(1)}&k_{42}^{(1)}&k_{43}^{(1)}&k_{44}^{(1)}%
\end{bmatrix}%
\]%
\[%
k_{1}= \begin{bmatrix}%
0.0&0.0&-0.0&-0.0\\%
0.0&163.6&-0.0&-163.6\\%
-0.0&-0.0&0.0&0.0\\%
-0.0&-163.6&0.0&163.6%
\end{bmatrix}%
\]%
Elemento {2}:%
\[%
k_{2}= \begin{bmatrix}%
k_{11}^{(2)}&k_{12}^{(2)}&k_{13}^{(2)}&k_{14}^{(2)}\\%
k_{21}^{(2)}&k_{22}^{(2)}&k_{23}^{(2)}&k_{24}^{(2)}\\%
k_{31}^{(2)}&k_{32}^{(2)}&k_{33}^{(2)}&k_{34}^{(2)}\\%
k_{41}^{(2)}&k_{42}^{(2)}&k_{43}^{(2)}&k_{44}^{(2)}%
\end{bmatrix}%
\]%
\[%
k_{2}= \begin{bmatrix}%
62.8&-47.1&-62.8&47.1\\%
-47.1&35.3&47.1&-35.3\\%
-62.8&47.1&62.8&-47.1\\%
47.1&-35.3&-47.1&35.3%
\end{bmatrix}%
\]%
Elemento {3}:%
\[%
k_{3}= \begin{bmatrix}%
k_{11}^{(3)}&k_{12}^{(3)}&k_{13}^{(3)}&k_{14}^{(3)}\\%
k_{21}^{(3)}&k_{22}^{(3)}&k_{23}^{(3)}&k_{24}^{(3)}\\%
k_{31}^{(3)}&k_{32}^{(3)}&k_{33}^{(3)}&k_{34}^{(3)}\\%
k_{41}^{(3)}&k_{42}^{(3)}&k_{43}^{(3)}&k_{44}^{(3)}%
\end{bmatrix}%
\]%
\[%
k_{3}= \begin{bmatrix}%
0.0&0.0&-0.0&-0.0\\%
0.0&163.6&-0.0&-163.6\\%
-0.0&-0.0&0.0&0.0\\%
-0.0&-163.6&0.0&163.6%
\end{bmatrix}%
\]%
Elemento {4}:%
\[%
k_{4}= \begin{bmatrix}%
k_{11}^{(4)}&k_{12}^{(4)}&k_{13}^{(4)}&k_{14}^{(4)}\\%
k_{21}^{(4)}&k_{22}^{(4)}&k_{23}^{(4)}&k_{24}^{(4)}\\%
k_{31}^{(4)}&k_{32}^{(4)}&k_{33}^{(4)}&k_{34}^{(4)}\\%
k_{41}^{(4)}&k_{42}^{(4)}&k_{43}^{(4)}&k_{44}^{(4)}%
\end{bmatrix}%
\]%
\[%
k_{4}= \begin{bmatrix}%
122.7&0.0&-122.7&0.0\\%
0.0&0.0&0.0&0.0\\%
-122.7&0.0&122.7&0.0\\%
0.0&0.0&0.0&0.0%
\end{bmatrix}%
\]

%
\subsection{Matriz de rigidez Global}%
\label{subsec:MatrizderigidezGlobal}%
\[%
K= \begin{bmatrix}%
0.0&0.0&0.0&0.0&0.0&0.0&0.0&0.0\\%
0.0&163.6&0.0&-163.6&0.0&0.0&0.0&0.0\\%
0.0&0.0&185.5&-47.1&-122.7&0.0&-62.8&47.1\\%
0.0&-163.6&-47.1&198.9&0.0&0.0&47.1&-35.3\\%
0.0&0.0&-122.7&0.0&122.7&0.0&0.0&0.0\\%
0.0&0.0&0.0&0.0&0.0&163.6&0.0&-163.6\\%
0.0&0.0&-62.8&47.1&0.0&0.0&62.8&-47.1\\%
0.0&0.0&47.1&-35.3&0.0&-163.6&-47.1&198.9%
\end{bmatrix}%
\]

%
\end{document}